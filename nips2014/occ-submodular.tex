
\documentclass{article} % For LaTeX2e
\usepackage{nips13submit_e,times}


% For figures
\usepackage{graphicx} % more modern
%\usepackage{epsfig} % less modern
\usepackage{caption}
\usepackage{subcaption}
\usepackage{amsmath,amsfonts,amsthm,amssymb}
\usepackage{listings}
\usepackage{tabularx}
\usepackage{multicol}
\usepackage{float}
\usepackage[protrusion=true,expansion=true]{microtype}
\usepackage{wrapfig}
\setlength{\emergencystretch}{3em}
\usepackage[numbers]{natbib}

% For algorithms
\usepackage[algoruled,vlined]{algorithm2e}
%\usepackage{algorithm}
\usepackage{algorithmic}

\usepackage{multicol}
\usepackage{comment}

% As of 2011, we use the hyperref package to produce hyperlinks in the
% resulting PDF.  If this breaks your system, please commend out the
% following usepackage line and replace \usepackage{icml2013} with
% \usepackage[nohyperref]{icml2013} above.
\usepackage{hyperref}

% Packages hyperref and algorithmic misbehave sometimes.  We can fix
% this with the following command.
\newcommand{\theHalgorithm}{\arabic{algorithm}}


\graphicspath{{figures/}}


\newcommand{\ie}{{\em i.e.,}~}
\newcommand{\eg}{{\em e.g.,}~}

\begingroup
    \makeatletter
    \@for\theoremstyle:=definition,remark,plain\do{%
        \expandafter\g@addto@macro\csname th@\theoremstyle\endcsname{%
            \addtolength\thm@preskip\parskip
            }%
        }
\endgroup
\newtheorem{dfn}{Definition}[section]
\newtheorem{thm}{Theorem}[section]
\newtheorem{cor}[thm]{Corollary}
\newtheorem{lem}[thm]{Lemma}
\newtheorem{prop}[thm]{Proposition}
\newtheorem{exmp}[thm]{Example}
\newtheorem{claim}{Claim}

\floatstyle{ruled}
\newfloat{program}{thp}{lop}
\floatname{program}{Program}

\newenvironment{denseitemize}{
\begin{itemize}[topsep=2pt, partopsep=0pt, leftmargin=1.5em]
  \setlength{\itemsep}{4pt}
  \setlength{\parskip}{0pt}
  \setlength{\parsep}{0pt}
}{\end{itemize}}

\newenvironment{packed_enum}{
\begin{enumerate}
  \setlength{\itemsep}{4pt}
  \setlength{\parskip}{0pt}
  \setlength{\parsep}{0pt}
}{\end{enumerate}}




\lstdefinelanguage{scala}{
  morekeywords={abstract,case,catch,class,def,%
    do,else,extends,false,final,finally,%
    for,if,implicit,import,match,mixin,%
    new,null,object,override,package,%
    private,protected,requires,return,sealed,%
    super,this,throw,trait,true,try,%
    type,val,var,while,with,yield},
  otherkeywords={=>,<-,<\%,<:,>:,\#,@},
  sensitive=true,
  morecomment=[l]{//},
  morecomment=[n]{/*}{*/},
  morestring=[b]",
  morestring=[b]',
%  numbers=left,
  morestring=[b]"""
}



% Commenting system
\newcommand{\Comments}{1}
\newcommand{\note}[2]{\ifnum\Comments=1\textcolor{#1}{#2}\fi}
\newcommand{\xinghao}[1]{\note{red}{[XP: #1]}}
\newcommand{\joey}[1]{\note{blue}{[JG: #1]}}
\newcommand{\stef}[1]{\note{green}{[SJ: #1]}}
\newcommand{\tb}[1]{\note{cyan}{[TB: #1]}}

\newcommand{\argmin}{\operatornamewithlimits{argmin}}






%% ---------------------------------------------------------
%% Terminology
\newcommand{\term}[1]{\textbf{#1}}


%% ---------------------------------------------------------
%% Citation/Reference commands
\newcommand{\citecf}[1]{(\cf, \cite{#1})}
\newcommand{\tableref}[1]{Table~\ref{#1}}
\newcommand{\figref}[1]{Figure~\ref{#1}}
\newcommand{\listref}[1]{Listing~\ref{#1}}

\newcommand{\eqnref}[1]{Eq.~(\ref{#1})}
\newcommand{\secref}[1]{Section~\ref{#1}}
\newcommand{\chapref}[1]{Chapter~\ref{#1}}

\newcommand{\dfnref}[1]{Definition~\ref{#1}}
\newcommand{\thmref}[1]{Theorem~\ref{#1}}
\newcommand{\propref}[1]{Prop.~\ref{#1}}
\newcommand{\lemref}[1]{Lemma~\ref{#1}}
\newcommand{\exmpref}[1]{Example~\ref{#1}}
\newcommand{\corref}[1]{Corollary~\ref{#1}}
\newcommand{\algref}[1]{Alg.~\ref{#1}}
\newcommand{\procref}[1]{Proc.~\ref{#1}}
\newcommand{\alglineref}[1]{Line~\ref{#1}}
\newcommand{\probref}[1]{Problem~(\ref{#1})}
\newcommand{\appendref}[1]{Appendix~\ref{#1}}

%% ---------------------------------------------------------
%% Basic Math
\newcommand{\abs}[1]{\left|#1\right|}
\newcommand{\ceil}[1]{\left\lceil #1 \right\rceil}
\newcommand{\floor}[1]{\left\lfloor #1 \right\rfloor}


%% ---------------------------------------------------------
%% special math functions
\newcommand{\polylog}[2]{\,\mathbf{Li}_{#1}\left( #2 \right)}
\newcommand{\harmonic}[2]{\,\mathbf{h}_{#1}\left( #2 \right)}

%% ---------------------------------------------------------
%% Norms
\newcommand{\Lone}{L_{1}}
\newcommand{\Linf}{L_{\infty}}
\newcommand{\LInfNorm}[1]{\left|\left| #1 \right|\right|_{\infty}}
\newcommand{\LOneNorm}[1]{\left|\left| #1 \right|\right|_1}



%% ---------------------------------------------------------
%% Probability notation
\newcommand{\given}{\,|\,}
\newcommand{\stdist}[1]{\mathbf{\pi} \left( #1 \right) }
\newcommand{\Prb}[1]{\mathbf{P} \left( #1 \right) }
\newcommand{\PrbEst}[1]{\mathbf{\tilde{P}} \left( #1 \right) }
\newcommand{\Ent}[1]{\mathbf{H} \left( #1 \right) }
\newcommand{\PiPrb}[1]{\Prb{ #1 } }
\newcommand{\Kern}[1]{K \left( #1 \right) }
\newcommand{\Ex}[1]{\mathbf{E} \left[ #1 \right] }
\newcommand{\Exwrt}[2]{\mathbf{E}_{#1} \left[ #2 \right] }
\newcommand{\Variance}[1]{\mathbf{Var} \left[ #1 \right] }
\newcommand{\Ind}[1]{\mathbf{1}\left[ #1 \right]}
\newcommand{\Bern}[1]{\text{Bern}( #1 ) }

%% ---------------------------------------------------------
%% Set notation
\newcommand{\reals}{\mathbb{R}}
\newcommand{\integers}{\mathbb{Z}}
\newcommand{\set}[1]{\left\{#1\right\}}
\newcommand{\vecspace}{\mathcal{V}}
\newcommand{\Union}{\bigcup}
\newcommand{\Inter}{\bigcap}
\newcommand{\union}{\cup}
\newcommand{\inter}{\cap}
\newcommand{\size}[1]{\left| #1 \right|}


%% ---------------------------------------------------------
%% Complexity
\newcommand{\BigO}[1]{O\hspace{-1pt}\left( #1 \right)}
\newcommand{\BigTheta}[1]{\Theta \left( #1 \right)}
\newcommand{\BigOmega}[1]{\Omega \left( #1 \right)}





%% ---------------------------------------------------------
%% Algorithms
\SetKwFor{ParForAll}{for}{do in parallel}{end}
\SetKwFunction{Map}{Map}
\SetKwFunction{Reduce}{Reduce}

\SetKwInput{Input}{Input}
\SetKwInput{Output}{Output}
\SetKwInput{SideEffect}{SideEffect}
\SetKwInput{Define}{Define}
\SetKwInput{Global}{Global}
\SetKwFor{DoWithProbability}{with probability}{}{}
\SetKwFunction{DPMeansOp}{DPMeansOp}
\SetKwFunction{DPValidate}{DPValidate}
\SetKwFunction{OFLValidate}{OFLValidate}
\SetKwFunction{BPMeansOp}{BPMeansOp}
\SetKwFunction{BPValidate}{BPValidate}
\SetKwFunction{NewClusters}{AcceptedClusters}

\SetKwFunction{Mean}{Mean}
\SetKwFunction{Ref}{Ref}


%% ---------------------------------------------------------
%% Paper specific notation

% All the data
\newcommand{\data}{\mathcal{D}}
% \datablock{machine}
\newcommand{\datablock}[1]{\data_{#1}}

\newcommand{\clusters}{\mathcal{C}}
\newcommand{\gclusters}{\hat\clusters}
\newcommand{\newclusters}{\tilde\clusters}
% local clusters \lclusters{machine}
\newcommand{\lclusters}[1]{\clusters_{#1}}

%\newcommand{\bregd}[2]{D_\phi\left(#1,#2\right)}
\newcommand{\bregd}[2]{\left\|#1-#2\right\|}


% for ofl analysis:
\newcommand{\CFL}{C^{\text{FL}}}
\newcommand{\muFL}{\mu^{\text{FL}}}


\title{(Optimistic?) Concurrency Control for \\ Non-monotone Submodular Maximization}

\author{
Xinghao Pan$^1$ Joseph Gonzalez$^1$ Stefanie Jegelka$^1$ Joseph Bradley$^{1}$ Michael I. Jordan$^{1,2}$\\
$^1$Department of Electrical Engineering and Computer Science, and $^2$Department of Statistics\\
University of California, Berkeley\\
Berkeley, CA USA 94720\\
  \texttt{\{xinghao,jegonzal,stefje,tab,?\}@eecs.berkeley.edu} \\
}

% \address{University of California
% 465 Soda Hall, MC-1776
% Berkeley, CA 94720-1776}

\newcommand{\fix}{\marginpar{FIX}}
\newcommand{\new}{\marginpar{NEW}}


%\nipsfinalcopy

\begin{document}


\maketitle


\begin{abstract}
Distributed non-monotone submodular maximization, serially equivalent to sequential algorithm.
\end{abstract}

\section{Introduction}

\section{Submodular maximization}

\section{Algorithm}
\subsection{Sequential}

\begin{figure}[h]
  \footnotesize
  \centering
  \begin{multicols}{2}
    \begin{minipage}{0.49\textwidth}
      \begin{algorithm}[H]
        \DontPrintSemicolon
        \caption{Serial submodular maximization}
        \label{alg:submax}
        %\Input{}
        $A^0 = \emptyset$, $B^0 = V$\;
        \For{$i = 1$ to $n$}{
          $\Delta_{+}(i) = [F(A^{i-1}\cup i) - F(A^{i-1})]_+$\;
          $\Delta_{-}(i) = [F(B^{i-1}\backslash i) - F(B^{i-1})]_+$\;
          Draw $u_i\sim Unif(0,1)$\;
          \If {$u_i<\frac{\Delta_{+}(i)}{\Delta_{+}(i)+\Delta_{-}(i)}$}{
            $A^i := A^{i-1} \cup i$;
            $B^i := B^{i-1}$\;
          }\Else{
            $A^i := A^{i-1}$;
            $B^i := B^{i-1}\backslash i$\;
          }
        }
        %\Output{$A_n$}
      \end{algorithm}
      \begin{algorithm}[H]
        \DontPrintSemicolon
        \caption{Validate($i$,$u_i$)}
        \label{alg:submax_master}
        \Input{$A^{i-1}$, $B^{i-1}$}
        $\Delta_{+}(i) = [F(A^{i-1}\cup i) - F(A^{i-1})]_+$\;
        $\Delta_{-}(i) = [F(B^{i-1}\backslash i) - F(B^{i-1})]_+$\;
        \If {$u_i<\frac{\Delta_{+}(i)}{\Delta_{+}(i)+\Delta_{-}(i)}$}{
          $A^i := A^{i-1} \cup i$;
          $B^i := B^{i-1}$\;
        }\Else{
          $A^i := A^{i-1}$;
          $B^i := B^{i-1}\backslash i$\;
        }
        %\Output{$A_n$}
      \end{algorithm}
    \end{minipage}

    \begin{minipage}{0.49\textwidth}
      \begin{algorithm}[H]
        \DontPrintSemicolon
        \caption{Parallel processing of element $i$}
        \label{alg:submax_worker}
        \Input{$A^j$, $B^j$, where $i'<i$}
        $C^{ji} = \{i'+1,\dots,i-1\}$\;
        $\Delta_{+}^{\max}(i) = [F(A^j\cup i) - F(A^j)]_+$\;
        $\Delta_{+}^{\min}(i) = [F(A^j\cup C^{ji}\cup i) - F(A^j\cup C^{ji})]_+$\;
        $\Delta_{-}^{\max}(i) = [F(B^j\backslash i) - F(B^j)]_+$\;
        $\Delta_{-}^{\min}(i) = [F(B^j\backslash C^{ji}\backslash i) - F(B^j\backslash C^{ji})]_+$\;
          Draw $u_i\sim Unif(0,1)$\;
          \If {$u_i<\frac{\Delta_{+}^{\min}(i)}{\Delta_{+}^{\min}(i)+\Delta_{-}^{\max}(i)}$}{
            $A^i := A^{i-1} \cup i$;
            $B^i := B^{i-1}$\;
          }\ElseIf{$u_i>\frac{\Delta_{+}^{\max}(i)}{\Delta_{+}^{\max}(i)+\Delta_{-}^{\min}(i)}$}{
            $A^i := A^{i-1}$;
            $B^i := B^{i-1}\backslash i$\;
          }\Else{Validate($i$,$u_i$)}
        %\Output{$A_i$, $B_i$}
      \end{algorithm}
    \end{minipage}
    
    
  \end{multicols}
  \caption{ 
The serial submodular maximization algorithm and distributed implementation. 
}
  \label{fig:submax}
\end{figure}

The sequential algorithm monotonically grows $A^i$ and shrinks $B^i$.

\subsection{Distributed}
We assume a total ordering on the elements, without loss of generality, let the ordering be $1,2,\dots,n$.
Let $j$ be such that $j<i$, and $C^{ji} = \{j+1,\dots,i-1\}$.
The properties of the (sequential) algorithm ensures that
\begin{align*}
A^j \subseteq &A^{i-1} \subseteq A^j\cup C^{ji},\\
B^j\backslash C^{ji} \subseteq &B^{i-1} \subseteq B^j.
\end{align*}
If we let $D^k = \{k': k+1 \leq k' \leq n\}$,
it is easy to see that $B^j = A^j \cup D^j = A^j \cup C^{ji} \cup i \cup D^i$ and $B^j\backslash C^{ji} = A^j \cup i \cup D^i$.
We define
\begin{align*}
\Delta_{+}^{\min}(i) &= [F(A^j\cup C^{ji}\cup i) - F(A^j\cup C^{ji})]_+\\
\Delta_{+}       (i) &= [F(A^{i-1}\cup i) - F(A^{i-1})]_+\\
\Delta_{+}^{\max}(i) &= [F(A^j\cup i) - F(A^j)]_+\\
\Delta_{-}^{\min}(i) &= [F(B^j\backslash C^{ji}\backslash i) - F(B^j\backslash C^{ji})]_+\\
                            &= [F(A^j\cup D^i) - F(A^j\cup D^i \cup i)]_+\\
\Delta_{-}       (i) &= [F(B^{i-1}\backslash i) - F(B^{i-1})]_+\\
                            &= [F(A^{i-1}\cup D^i) - F(A^{i-1}\cup D^i \cup i)]_+\\
\Delta_{-}^{\max}(i) &= [F(B^j\backslash i) - F(B^j)]_+\\
                            &= [F(A^j\cup D^j\backslash i) - F(A^j\cup D^j)]_+\\
                            &= [F(A^j\cup C^{ji}\cup D^i) - F(A^j\cup C^{ji}\cup D^i \cup i)]_+
\end{align*}
Submodularity of $F$ implies that
\begin{align*}
\Delta_{+}^{\min}(i) &\quad\leq\quad \Delta_{+}(i) \quad\leq\quad \Delta_{+}^{\max}(i),\\
\Delta_{-}^{\min}(i) &\quad\leq\quad \Delta_{-}(i) \quad\leq\quad \Delta_{-}^{\max}(i),
\end{align*}
so we can bound
\begin{equation*}
lb(i) ~~:=~~
\frac{\Delta_{+}^{\min}(i)}{\Delta_{+}^{\min}(i) + \Delta_{-}^{\max}(i)}
\quad\leq\quad 
\frac{\Delta_{+}(i)}{\Delta_{+}(i) + \Delta_{-}(i)}
\quad\leq\quad 
\frac{\Delta_{+}^{\max}(i)}{\Delta_{+}^{\max}(i) + \Delta_{-}^{\min}(i)}
~~=:~~ ub(i).
\end{equation*}
Thus, if $u_i \leq lb(i)$, we can safely grow $A^i = A^{i-1} \cup i$.
Conversely, if $u_i \geq ub(i)$, we can safely shrink $B^i = B^{i-1} \backslash i$.
(In practice, growing $A$ or shrinking $B$ is done implicitly by setting an indicator variable.)

\section{Distributed function computation}
To make our concurrent submodular maximization algorithm work, it is essential that the computation of $\Delta$'s can be done efficiently.
We maintain a \textit{sketch} for $A$, which contains all essential information for computing $\Delta$'s, and update the sketch as elements are added to $A$.
Similarly, we maintain sketches for $D^i$'s; since the sets $D^i$'s are known at the start of the algorithm, we can pre-compute the sketches in advance.

We'll show in the examples below that updating the sketch for $A$ and pre-computing the sketch for $D$ can be done efficiently, and that we can compute the $\Delta$'s easily from our sketches.

\xinghao{See Stefanie's write-up for greater exposition on sketching functions}

\subsection{Graph cut}
$F(A) = \sum_{i\in A}\sum_{j\in V\backslash A, (i,j)\in E} w(i,j)$.
The sketch for $A$ is simply $A$ itself, and is maintained by each processor.
The sketch for $D^j$ is the single number $j$.

\subsection{Separable sums}

$F(A) = \sum_{l=1}^k g\left(\sum_{i\in A\cap S_l}w(i)\right) - \lambda\sum_{i\in A}v(i)$.
The sketch for $A$ is the $k+1$ vector containing the sums $\sum_{i\in A\cap S_l}w(i)$ and $\sum_{i\in A}v(i)$.
Similarly for $D$.
Updating $A$ involves adding $w(i)$ and $v(i)$ to the sums for $A$.
Pre-computing the sketch for $D$ requires computing $k+1$ cumulative sums, one for each $S_l$, of length $N$.


\section{Upper bound on expected number of elements sent for validation}
Let $N$ be the number of elements, i.e. the cardinality of the ground set.
Let $P$ be the number of processors.

We assume that the total ordering assigns elements to processors in a round robin fashion.
Thus, we assume $C^{ji}=\{i-p+1,\dots,i-1\}$ has $p-1$ elements.

We call element $i$ \textit{dependent} on $i'$ if $\exists A, F(A\cup i)-F(A) \neq F(A\cup i' \cup i)-F(A\cup i')$ or $\exists B, F(B\backslash i)-F(B) \neq F(B\cup i'\backslash i) - F(B\cup i')$, i.e. the result of the transaction on $i'$ will affect the computation of $\Delta$'s for $i$.
For example, for the graph cut problem, every vertex is dependent on its neighbors; for the separable sums problem, $i$ is dependent on $\{i': \exists S_l, i\in S_l, i'\in S_l\}$.

Let $n_i$ be the number of elements that $i$ is dependent on.

Now, we note that if $C^{ji}$ does not contain any elements on which $i$ is dependent, then $\Delta_{+}^\text{max}(i) = \Delta_{+}(i) = \Delta_{+}^\text{min}(i)$ and $\Delta_{-}^\text{max}(i) = \Delta_{-}(i) = \Delta_{-}^\text{min}(i)$, so $i$ will not be validated (in either deterministic or probabilistic versions).
Conversely, if $i$ is validated, there must be some element $i'\in C^{ji}$ such that $i$ is dependent on $i'$.

\begin{align*}
&E(\text{number of validated elements})\\
=& \sum_i P(i \text{ validated})\\
\leq& \sum_i P(\exists i'\in C^{ji}, i \text{ depends on } i')\\
=& \sum_i 1-P(\forall i'\in C^{ji}, i \text{ does not depend on } i')\\
=& \sum_i 1-\prod_{k=1}^{P-1}\frac{N-k-n_i}{N-k}\\
=& \sum_i 1-\prod_{k=1}^{P-1}\left(1-\frac{n_i}{N-k}\right)\\
\leq& \sum_i 1-\left(1-\sum_{k=1}^{P-1}\frac{n_i}{N-k}\right) & \text{(Weierstrass inequality)}\\
=& \left(\sum_i n_i\right)\left(\sum_{k=1}^{P-1}\frac{1}{N-k}\right)\\
\leq& \frac{P-1}{N-P+1}\sum_i n_i.
\end{align*}

The key quantity in the above inequality is $\sum_i n_i$.
Typically, we expect each element $i$ to depend on a small fraction of the ground set.
For example, in the graph cut problem, $\sum_i n_i = 2|E|$ is twice the number of edges.
If the graph is sparse with $|E|\approx s|V|\log|V|$, where $0\leq s\ll 1$ and $P\ll N$, then $\frac{P-1}{N-P+1}\sum_i n_i \approx 2s(P-1)\log N$, which grows sublinearly with $N$.

Note that the bound established above is generic to all algorithms that follow the basic transactional model we proposed (round-robin optimistic concurrency control), and is not specific to $F$ or even submodular maximization.
Thus, while our bounds provide a fundamental limit, additional knowledge of $F$ can lead to better analyses on the algorithm's concurrency.




\subsection{Tighter general bound?}
Define $\rho_i = \max_{S\subseteq V} \{[F(S\cup i) - F(S)] - [F(S \cup C^{ji} \cup i) - F(S \cup C^{ji})]\} \leq F(i) - F(V) + F(V\backslash i)$

\xinghao{Is there theory along these lines?}

Then, we can bound
\begin{align*}
\Delta_+^{\min} \leq \Delta_+^{\max} \leq \Delta_+^{\min} + \rho_i && \text{(choosing $S=A^j$)}\\
\Delta_-^{\min} \leq \Delta_-^{\max} \leq \Delta_-^{\min} + \rho_i && \text{(choosing $S=A^j\cup D^i$)}
\end{align*}

Thus,
\begin{align*}
&E(\text{number of validated elements})\\
=& \sum_i P(i \text{ validated})\\
=& \sum_i P\left(\frac{\Delta_+^{\max}}{\Delta_+^{\max} + \Delta_-^{\min}} \leq u_i \leq \frac{\Delta_+^{\min}}{\Delta_+^{\min} + \Delta_-^{\max}}\right)\\
=& \sum_i\frac{\Delta_+^{\max}}{\Delta_+^{\max} + \Delta_-^{\min}} - \frac{\Delta_+^{\min}}{\Delta_+^{\min} + \Delta_-^{\max}}\\
\leq& \sum_i\frac{\Delta_+^{\min}+\rho_i}{\Delta_+^{\min} + \rho_i + \Delta_-^{\min}} - \frac{\Delta_+^{\min}}{\Delta_+^{\min} + \rho_i + \Delta_-^{\min}}\\
=& \sum_i\frac{\rho_i}{\Delta_+^{\min} + \rho_i + \Delta_-^{\min}}
\end{align*}




\subsection{Upper bound for max graph cut}
Denote $\tilde{A}^j = V\backslash A^j\backslash C^{ji}\backslash D^i = \{1,\dots,j\}\backslash A^j$ be the elements up to $j$ that are not included in $A$.
Let $w_i(S) = \sum_{j\in S, (i,j)\in E} w(i,j)$.
For the max graph cut function, it is easy to see that 
\begin{align*}
\Delta_+^{\min} &= \max(0, - w_i(A^j) -w_i(C^{ji}) + w_i(D^i) + w_i(\tilde{A}^j))\\
\Delta_+^{\max} &= \max(0, - w_i(A^j) + w_i(C^{ji}) + w_i(D^i) + w_i(\tilde{A}^j))\\
\Delta_-^{\min} &= \max(0, + w_i(A^j) - w_i(C^{ji}) + w_i(D^i) - w_i(\tilde{A}^j))\\
\Delta_-^{\max} &= \max(0, + w_i(A^j) + w_i(C^{ji}) + w_i(D^i) - w_i(\tilde{A}^j))
\end{align*}

Consider the following cases.
\begin{itemize}
\item $\Delta_+^{\max} = 0$. Then $\Delta_+^{\min} = 0$ and also
\begin{align*}
w_i(A^j) &> w_i(C^{ji})+ w_i(D^i) + w_i(\tilde{A}^j)
\quad\implies\quad w_i(A^j) + w_i(D^i) > w_i(C^{ji}) + w_i(\tilde{A}^j)
\end{align*}
so $\Delta_-^{\min} > 0$ and $\Delta_-^{\max}>0$.
Thus $\frac{\Delta_+^{\max}}{\Delta_+^{\max} + \Delta_-^{\min}} - \frac{\Delta_+^{\min}}{\Delta_+^{\min} + \Delta_-^{\max}} = 0-0 = 0$.

\item $\Delta_-^{\max} = 0$. Then $\Delta_-^{\min} = 0$ and also
\begin{align*}
w_i(\tilde{A}^j) &> w_i(C^{ji})+ w_i(D^i) + w_i(A^j)
\quad\implies\quad w_i(\tilde{A}^j) + w_i(D^i) > w_i(C^{ji}) + w_i(A^j)
\end{align*}
so $\Delta_+^{\min} > 0$ and $\Delta_+^{\max} > 0$.
Thus $\frac{\Delta_+^{\max}}{\Delta_+^{\max} + \Delta_-^{\min}} - \frac{\Delta_+^{\min}}{\Delta_+^{\min} + \Delta_-^{\max}} = 1-1 = 0$.

\item $\Delta_+^{\max}>0$ and $\Delta_-^{\max}>0$.
Then,
\begin{align*}
&\frac{\Delta_+^{\max}}{\Delta_+^{\max} + \Delta_-^{\min}} - \frac{\Delta_+^{\min}}{\Delta_+^{\min} + \Delta_-^{\max}} \\
=& \frac{- w_i(A^j) + w_i(C^{ji}) + w_i(D^i) + w_i(\tilde{A}^j)}{- w_i(A^j) + w_i(C^{ji}) + w_i(D^i) + w_i(\tilde{A}^j) + \max(0, + w_i(A^j) - w_i(C^{ji}) + w_i(D^i) - w_i(\tilde{A}^j))}\\
 &- \frac{\max(0, - w_i(A^j) -w_i(C^{ji}) + w_i(D^i) + w_i(\tilde{A}^j))}{\max(0, - w_i(A^j) -w_i(C^{ji}) + w_i(D^i) + w_i(\tilde{A}^j)) + w_i(A^j) + w_i(C^{ji}) + w_i(D^i) - w_i(\tilde{A}^j)}\\
=& \min\left(1,\frac{ - w_i(A^j) + w_i(C^{ji}) + w_i(D^i) + w_i(\tilde{A}^j)}{2w_i(D^i)}\right)\\
 & -\max\left(0,\frac{- w_i(A^j) - w_i(C^{ji}) + w_i(D^i) + w_i(\tilde{A}^j)}{2w_i(D^i)}\right)\\
=& \min\left(1,\frac{ w_i(C^{ji})}{w_i(D^i)}\right)
\end{align*}
\end{itemize}

Thus,
\begin{align*}
E(\text{\# of validated elements})
= \sum_i\frac{\Delta_+^{\max}}{\Delta_+^{\max} + \Delta_-^{\min}} - \frac{\Delta_+^{\min}}{\Delta_+^{\min} + \Delta_-^{\max}}
\leq \sum_i \min\left(1,\frac{ w_i(C^{ji})}{w_i(D^i)}\right)
\end{align*}

\xinghao{Not sure how to sum this over $i$.}





{\footnotesize
%\subsection*{Acknowledgments}
%This research is supported in part by NSF CISE Expeditions award CCF-1139158 and DARPA XData Award FA8750-12-2-0331, and  gifts from Amazon Web Services, Google, SAP,  Blue Goji, Cisco, Clearstory Data, Cloudera, Ericsson, Facebook, General Electric, Hortonworks, Intel, Microsoft, NetApp, Oracle, Samsung, Splunk, VMware and Yahoo!.
%This material is also based upon work supported in part by the Office of
%Naval Research under contract/grant number N00014-11-1-0688. 
%X. Pan's work is also supported in part by a DSO National Laboratories Postgraduate Scholarship.

% In the unusual situation where you want a paper to appear in the
% references without citing it in the main text, use \nocite
%\nocite{langley00}

\bibliographystyle{unsrtnat}
\bibliography{references_arxiv}
}

%\newpage
%\appendix
%\input{appendix}



\end{document}


% This document was modified from the file originally made available by
% Pat Langley and Andrea Danyluk for ICML-2K. This version was
% created by Lise Getoor and Tobias Scheffer, it was slightly modified
% from the 2010 version by Thorsten Joachims & Johannes Fuernkranz,
% slightly modified from the 2009 version by Kiri Wagstaff and
% Sam Roweis's 2008 version, which is slightly modified from
% Prasad Tadepalli's 2007 version which is a lightly
% changed version of the previous year's version by Andrew Moore,
% which was in turn edited from those of Kristian Kersting and
% Codrina Lauth. Alex Smola contributed to the algorithmic style files.
